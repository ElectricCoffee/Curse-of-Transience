\section{Trolds -- Woodland Trolls}
Woodland trolls---often simply called \textit{Trolds}---hail from the colder regions of the world and have lived peacefully in societies parallel to humans for centuries.
It is not uncommon for Woodland trolls to be found living and working in human cities, but tend to keep to burrow-communities in the various forests.

\paragraph{Physical Description}
Trolds are relatively short for trolls, adults typically ranging from around 1.3m to around 1.6m.
Trolds have tails, long enough to reach the floor from a standing position, ending in a tuft of hair.
Like humans, trolls have five fingers on each hand and five toes on each foot, though unlike humans they have long pointy ears and small horns purtruding through their hair.
Troll body hair is also thicker than that found on humans.

Trolds have fair skin and thus easily get sunburnt, though unlike their mountainous counterparts (see~Section~\ref{sec:trolls}), they do not turn to stone in direct sunlight.
And while not blinded by direct sunlight, they do possess quite good night-vision.

Life-expectancy for a trold is around 130--240 years.

\paragraph{Society}
Trold usually live in family burrows spread across the various forests across the world.
Burrows usually have a population of 15-30 individuals, typically lead by the oldest matriarch or matriarchs in the burrow. 
This matriarch is referred to as \textit{Mother}.
The largest burrow recorded in Saoghall history has had 97 trolls living within.
As such, trolds do not usually create \textit{troll countries}, but rather try to integrate---if possible---into the existing surrounding societies.

The vast majority of trolds commit to a ceremonial topside journey---called \textit{The Exodus}---from their burrow, usually as a coming-of-age period or as a means of bringing in new skills and trades into the burrow.
Most trolds return to their burrows after a few years, but some choose to stay topside for the remainder of their lives.
The purpose of this exodus is threefold; it can thin out the population of a burrow so it doesn't become overcrowded, it brings new blood into the burrow by having the trolds return with a mate, or it brings in new ways and means into the burrow by having trolds learn new trades topside.

Just about every burrow in the world has an exodus, but this is usually motivated differently for each.

\paragraph{Relations}
While most trold burrows are isolated by their nature, their relation to the outside world varies wildly from friendly, to cautious, to outright xenophobic.

The practice of The Exodus is known to have caused problems within certain human societies, as the the influx ``outsiders'' to human villages and cities sometimes causes them receive blame for preceived economic pressure.

\paragraph{View on the Undead}
Trolds are generally uncomfortable with undeath.
Those living topside typically have a begrudging tolerance towards undeath, and will deal with it if need be, but those living in burrows are typically less keen on it.
Most burrows are not happy about trolds returning from The Exodus in a state of visible undeath, and---if signs are clear enough---will outright refuse the trold to return.

\paragraph{Religion}
Most trolds will worship the spirits of their local forest, generally focusing on a guardian figure local to their climate, though trolds born and raised in cities will sometimes adopt the local religion.

\paragraph{Adventurers}
As a natural consequence of The Exodus, many trolds choose to become adventurers to broaden their world view and acquire new skills.

\paragraph{Names}
Trolls have a relatively strict and complex naming convention.

The general pattern is \textsc{[first-name]}, \textsc{[occupation]} of \textsc{[burrow-name]}. \\For example: \textit{Janice, Tailor of Wierwood Warren}.

However, if a Trold has not yet gone on their Exodus, they typically use their parent's occupation and their relation to them:\\
For example: \textit{Ghorm, Fletcherson of Smokey Woods Hollow}.

Note that being the elder of a burrow is considered an occupation, thus \textit{Mother}, \textit{Father}, or \textit{Elder} are used in names.
This leads to situations where there are Trolds with names like \textit{Frigg, Mother of Deepmire Mott} and \textit{Harold, Motherson of Cape Languid Pits}.

\subsection{Racial Traits}

\paragraph{Night-Vision}