\section{The Religions of Saoghaal}
\subsection{The Three-Lakes Region}
The Triduchy is home to an old and complex pantheon of variously powerful Patron Deities. 
This religion has been conglomerated over time from the smaller cults of the original tribes that now make up the Peoples of the Triduchy.

\subsubsection{Primary Deities}

\paragraph{Kara, The Goddess of life and Source of all Life on the Planet}
She lays trapped at the centre of the planet, crushed beneath the all the rocks that make up the world and stabbed by enormous stalactites. Her blood is the purest of water, and the Lake of Kara is where it leaks to the surface unadulterated. Someday she will find the strength to break free of her prison, the world will be destroyed and she will hunt the cosmos for any of those who have drank of her blood and consume them to reclaim it, down to the most infinitely removed ancestor, this task will take her until the end of time, after which she will create a new and better world. It is known that drinking the water of Lake Kara is a willful pact, granting worldly power of life against hardship in exchange for an inescapable curse. Kara’s name is rarely called upon, and only under great stress as a superlative.

\paragraph{Mei, The Patron Spirit of Guardianship, Retribution, and Artistry}
Depicted as a formless shadow, Mei is often called upon for favor in acts of revenge, or begged for protection before battle, but Mei is a fickle and unknowable presence granting favor and judgement with a logic no mortal can comprehend. Stories tell of all light fading during struggles of great historical importance as this is the portent of Mei. Interestingly, many other stories also describe this portent coming over artists and craftsmen as they complete great works, placing powerful artistic endeavours under Mai’s domain as well. Sacrifices of blood and fat are burned in the name of Mei, but only when desperate as it is understood that calling Mei’s attention can be as dangerous as it is helpful. It is known that the most proper way to bring an end to one’s life is to bring one’s soul to Mei in the afterlife.

\paragraph{Kos is Dead, His Carcass Dissolved Into the Lake Named After Him}
Eons before the rise of mortals The Goddess Kara and her cohort Kos lived amicably. They were friendly rivals of creation, each with their own methods. Kos would make dirt, so Kara could make flowers. Kos devised the sea, thus Kara devised fish. Kos thought up sun and empty sky, Kara created birds. But this partnership soon became rivalry for Kos and his creations became more and more incompatible with Kara’s vision, Salt, Brimstone, Lava, and Lightning. Kara was pushed until she begged of Kos to cease his destructiveness. Thus Kos struck The Goddess down in anger, but Kara was not so easily quelled, and together they created Wrath. Legends vary wildly, and Kos himself is described as everything from some chimeric beast to a godly man not unlike Kara’s womanhood, but in their battle the great desert was carved from the lush world that preceded it and Kos and Kara had all but murdered each other. Where their ichor spattered to the ground and mixed together the animals and living creatures of the world sprang to life, having been accidentally gifted a will of their own by the careless violence of the gods. In a final desperate blow Kara tore our Kos’ heart and threw it on the ground, but to her horror Kos was so enraged that he lived on, furiously pummelling Kara until she was driven to the core of the planet. Kos’ victory was short lived, as his efforts had drowned the landscape in a great flood of both of the god’s blood, which promptly coagulated into Mei, who’s first primal act of birth was to devour the wounded Kos and then vomit him up, chewed and ruined, into the empty basin that now makes up Lake Kos.

\subsubsection{Lesser Deities}
\paragraph{Ond, The Progenitor Ember, A star Who Came too Close to the World}
Often stories say he was trying to catch a glimpse of a beautiful maiden, but when he got too close he hit a cloud and broke up in a shower of sparks and fire. This introduced embers to the world, without which there could be no starting of fires or lighting of forges. Ond is traditionally given thanks by smiths when they light their smithys, and praised when a campfire is lit. Some cooks consider themselves in thanks to Ond as well, but this is also sometimes said mockingly when food has been burned.

\paragraph{Yora, the First Woman}
After killing Kos, Mei took a time to rest and recuperate. Mei picked at their teeth where their gums has been cut by the chitin of Kos, and from the tiny speck of mixed god blood sprang Yora, who was flung down to the ground off of Mei’s finger. Yora looked around at the ruination surrounding her and knew that this terrific thing lounging beside her needed to be put to rest so life could survive in its shadow. Yora displayed her alluring beauty and attracted the wind, which gave Mei a chill, so the formless god allowed itself to lay flat in the basin it was resting in. Thus Yora climbed down into the basin and swam in the formless dark of Mei’s body. She gave solace to Mei, and from the sleeping God sprung a luminescent flood which filled the basin and blanketed Mei in quiet. Yora Eventually escaped the sleeping god’s body, and swam up from the bottom of Lake Mei. When she finally arrived at the shore Yora was pregnant with three sons, one for each of the years she spent inside Mei. She is called to for favor when pregnant or giving birth, and begged for strength by stressed mothers.

\paragraph{Geis, the First Man}
Shortly after making it to the shore of the newly formed Lake Mei, Yora fell into labor. Her first son was Geis, and after being born, he set himself to helping his mother give birth to his brothers. He brought water from the lake, and reeds from the shore. But as Geis washed his newborn brothers and laid them on the beds he’d made, they were poisoned by the water, and sliced painfully by the reeds. Angrily, Yora cast him away. Meaning never to disobey or do wrong by his mother, Geis accepted banishment without question. Geis is considered to have patronage over lost causes, tarnished honor, and unrequited love.

\paragraph{The Grim Siblings}
Yora’s younger sons were born together as twins, they were both of them perfect and beautiful. In his misguided efforts to comfort the babes, Geis inadvertently doomed the twins. Laying in a bed of razor-sharp reeds, their flesh was cut to the bone. Washed with the poisoned glowing waters of Lake Mei, they were cursed. Yora cast Geis away, and attempted to tend to the wounds of her remaining children. But she found that this task was beyond her. In desperation she swaddled up the shreds of her babies and sought the aid of each living creature on the planet. After many long journeys she eventually returned to Lake Mei, and in desperation Yora took her sons into the waters, seeking the aid of Mei itself. The God would only be able to save one of the sons, and the cost would be the other. But when Yara undid the swaddling around her twin sons she found that they had cleverly mixed their shredded bodies and souls, so that neither could be differentiated in any way. The Grim Siblings had hoped to outwit Mei and force a decision that favoured their mother. But the agreement settled upon between the First Woman and the God was to weigh out half of the bundle of flesh for each. Mei and Yora each would have many reasons to reconvene and reweigh half of the Grim Siblings, each time attempting to return the twins to some more whole state, never succeeding. The Grim Siblings share existence in a state of neither life nor death, each of the pair halfway in the real world, and halfway deep in the bosom of Mei. Neither truly knowing which of the pair they are anymore. One brother would prefer the solace of true death, to be wholly consumed by Mei and exist no more, he is depicted as calm and penitent. The less rational brother craves life and the surface world, often acting madly to get it and then maddened by having it. The Grim Siblings feature prominently in many stories, usually as protagonists. They represent Humanity as a whole in myth, and their dual nature of life and death is a metaphor for that of reality. In stories the Sane Brother is often depicted as having at least intentions which are noble, when he is in ascendancy. He is usually rational and polite. While the Mad Brother ranges from cautionary anti-hero, to comedic foil and rube. This brother is usually extremely uncouth and rude.

\paragraph{Huf, the Small Wind}
Long ago, a tiny wind got lost and ended up on the far side of the world. Eventually this Wind discovered the three lakes, and as the wind spoke in airs rather than words, Huf assumed that Kos was the most talkative. Huf had spent eons alone, traversing the world and seeing sights, and so Huf took great pleasure in regaling Kos’ corpse with them. Over time Huf developed a one sided love for the corpse of Kos and chose to lay with it. Huf still tells stories to Kos and these tiny windy words bubble to the surface and make foam. Most of the stories Huf tells are crude humor, or some fascinated description of a stench now added to the repertoire of Lake Kos. But one of the stories is from across the world, and if Huf had not brought it to Kos, People would be without bread and beer. This story is the scent of leavening, and where the words of Huf bubble to the surface, one can carefully skim the resultant foamy scum. If one is lucky, it is possible cleanse the smell Huf described of all the remnant stenches that make up Lake Kos’ waters, and use the Leavening from Huf’s story to make bread or beer. But this is no simple matter, and expertise is required to misuse even the power of such a small god in this way.

\paragraph{Fuug the Relentless Wind}
Many winds both great and small have names and stories, but among them none compare to Fuug. Depicted as a cloud or a feather, or simply evoked in the ferocity of a wind, Fuug is the wind against which all others are judged, and the one which gusts when all others fade. Before the world was formed, there was only the sky and the sun, Fuug was born of the very will of the sky to hold back the fires of the sun. Without Fuug nothing could exist, as all would be incinerated were the sun not kept aloft in the sky. Though beneficial to humanity, Fuug is not a benevolent presence. In the battle against the sun, Fuug if often the cause of wanton earthly destruction. Often the power of Fuug is not simply in a wind that is blowing fast, but in a wind that blows for days and days, months or even years. Fuug is where sandstorms come from, and the very weather is beholden to the ramifications of Fuug’s war. Fuug is the cosmic source of determination and all mortals who exhibit a strong force of will have channelled it from Fuug. But the unfortunate counterbalance to this is that when Fuug has a great need of strength, it will be tapped and sapped away from mortals instead. When a person cannot be stopped, this is because they have gained strength through Fuug, but should that person become overcome with dread and ennui this too is Fuug’s doing. Such is the transaction that allows mortals to have force of will at all. The name of Fuug is uttered in a call for personal strength, usually by men when committing physical feats. Fuug is also thanked, both sincerely and in jest, for the state of the weather, or occasionally for changes of circumstances out of anyone’s control.

\paragraph{Hy, The Fidget-Sprite -- A Being From Beyond This Reality who Came to the World for Reasons Unknown}
He gifts mortals with ideas, which he expects them to turn into a reality. This can take the form of a sudden and obvious clarity on how to complete a minor chore, or unstoppable visions of insane machinations that are as impossible to hew into literal form as they are haunting your vision even with closed eyes. Hyr is credited with gifting humanity with such things as the weaving and dyeing of textiles, and alchemy. Offerings to Hyr take to form of sharing knowledge, the lowest form being gifting scrolls or carvings and the highest form being the taking of an apprentice. It is considered a more positive offering to Hyr to share useful or entertaining knowledge. The opposite act, sharing terrible or damaging knowledge is also a way to gain the favor of Hyr, but is frowned upon by those in society who do not share that inclination.

\paragraph{The Deafening Silence of Het}
All around the three lakes region, for as long as people have told stories, they have told variations on the story of Het. There is much variance, Het often being a shadow like Mei, or a man with flesh that looked like the night sky, or sometimes simply the realm featured in the story with no personification at all. But what all the stories share is the following series of events; A person is put in a desperate and noisy situation, sometimes this is one of the other gods, or other times just a mortal. They call out rhetorically for silence, and they find their call receives a response. A deafening bell is run out from nowhere and everywhere, and whoever made the error of calling upon Het is taken to a place of utter deep impenetrable dark silence from which there is no escape. Various telling of the tale branch off from here with moralizing and often a returning of the victim to the world, but just as many serve as a stark warning of the dangers of consequence.

\paragraph{Kerber, the Traitorous Beast}
One of the older myths of the Three-Lakes Region speaks about Kerber and how he worked with the Grim Sibling. Different tellings feature more or less of the different Siblings, and alter how exploitative and by which side of the pact the situation is. But all variations tell of how the Grim Sibling found himself stymied by a natural impasse, usually rocks blocking a path, and that they procured the aid of Kerber to clear it. Kerber is generally depicted as a Giant Horned-Lizard. Kerber is called a traitor, for gifting mankind the power to domesticate him, and by extension all animals. Thus Kerber betrayed the beasts of the world, whether willingly or not.

\paragraph{Irso, The First Mother -- The Subject of Very Old Myths}
During their earliest journeys across the surface world, the Grimm Siblings discovered the one thing they agreed upon, that this realm was incredibly lonely. The story is told in many different ways, suggesting that the Grimm Siblings were trying to recreate Yora, or saw inspiration in the families of the animals around them. Regardless of why, the stories venture that the Grim Siblings worked together over a very long time to construct Irso. Made of small parts of themselves, and traits of noble creatures taken as trophies. Irso was given the spark of life with a washing of the Waters of Kara, but she remained incomplete. She possessed no feet and no eyes, for the Grimm Siblings had none to spare and found none to use. With her first breath she asked to be taken toward a soft grass, and from it she wove a blanket, which she shared with the Grim Siblings. Irso is characterized as almost naively caring, and entirely unable to tell the difference between the Grim Siblings. She fills the role in myth of Mother, on occasion caring for the Grim Siblings, or raising their children. Irso rivals Yora in a shared role of being worshipped when expecting children. But her name is rarely uttered as an expletive in the way Yora’s is. On occasion a mother or caring individual will be equated to Irso as a sign of acknowledgment of an act of great kindness. Some people consider Irso the patron of the downtrodden or even the undead, while others consider this association derogatory of her.

\paragraph{The Many}
Around the Triduchy there are various legends of a creeping multitudinous deity that infests the desert. Some peoples consider The Many to be comprised of all crawling insects, while other peoples believe The Many consists exclusively of Bloommound-Mites. Legends are not clear on the origin of The Many, but generally describe it as infesting the damaged and unhealthy part of the world that is the great desert. The Many possesses a body that is the insects as a collective, a greater whole than the sum of the individuals. In art the Many is almost always depicted as a colossal Bloommound Mite, borne on a swarm of other mites. The Many has aspirations that cannot be understood by mortals, and resides underground where all Bloommounds connect together in a great infested warren. The collecting of clay from these mounds is commonplace, but doing so is known to be in defiance of The Many, and offerings of food for the mound when taking clay are common, as are practices of allowing the mites to exit the selected clay lumps. Others are more willing to embrace the adversarial nature of this relationship, and specifically choose less humane methods of acquiring the clay. Depictions of The Many, or aspects therein, are commonplace on pottery in the region.