\subsection{The Triduchy of Karakosmei -- The Duchies of the Desert}
\begin{tabular}{r | l}
    \textbf{Capital} & The Duchy of Kara\\
    \textbf{Total Population} & \\
    \textbf{National Language} & 
\end{tabular}

The Triduchy of Kara, Kos, and Mei is a land of three co-dependant Duchies under the de-facto rule of the Duke of Kara, who is the leader of the state religion of the Triduchy. 
The Three-Lakes Region was historically home to a multitude of tribes and peoples, but is now a densely populated metropolis, yet there remains a tapestry of overlapping traditions and beliefs. 
The three Duchies are named for the three lakes around which they have been settled.

Lake Kara is sourced by a vast underground aquifer, and it’s pristine waters spill out to feed its neighboring lakes. 
The settlements on the shores of Kara are home to the rich and powerful.

Lake Mei is to the east of Lake Kara, and it’s still waters are home to a unique and beautiful algae. 
This algae has seasonal luminescent blooms, which contaminate the lake-water with Cyanide.

Lake Kos sits to the west of Lake Kara, the locals make regular thanks to their gods that this places Lake Kos downwind of the rest of the Triduchy. 
The Waters of Kos are opaque, vile, and deadly. 
One must protect their skin from exposure to it. 
Additionally should one somehow manage to swill some of it in disregard for the stench, a painful death would shortly follow. 
The ground has steeped Arsenic and heavy-metal salts into the water, crystals of the poison can be collected from it’s shores. 
While alchemical dredging is performed to reap precious metals from the leachate.

\subsubsection{Economy}
The region exports precious metals, dyes, textiles, and art-pieces of all description. 
While importing large quantities of grain and other smaller quantities of higher-desirability foodstuffs. 
Internally the Duchy of Kos consumes the most grain, followed by the Duchy of Mei and then the Duchy of Kara after that. 
While the opposite trend is true for imported foodstuffs such as fruits, vegetables, and meats. 
The Duchy of Kos is where precious metals are leached from the lake and processed into usable materials, these metals are then sold into the Duchy of Mei to be worked into more valuable wares. 
The Duchy of Kos produces the majority of the region’s bricks and almost all of the pottery-grade clay is processed there. 
Seasonally Lake Kos has fishing booms and temporary economies will thrive around the capture and processing of Kostic Eels, though most of this meat stays within the Duchy of Kos. 
Foraging in the desert surrounding the Triduchy is an important trade, and where much local food as well as all local clay and wood is sourced.

\subsubsection{Culture}
The peoples of the Triduchy are varied but share a will to survive in the great desert as a collective. 
Individually they strive to see past their many differences. 
The Triduchy nonetheless has an extremely classist and segregated society. 
They share a rich pantheonic religion with the region being home to many different cults. 
Hard work is valued, as is a visible effort to fit a niche in the community.
Undeath is frowned upon, and the undead are often forced to hide their nature, usually only able to work the lowest of trades in the Duchy of Kos. 
The people believe that the proper resolution to death is to journey through Niflhel to Lake Mei and seek solace with the formless deity. 
In the Three-Lakes Region the peoples’ gods are the winds and the lakes around them, their gods are tactile and very real to them. 
Natural phenomena are personified, winds named, and mortals are not considered the fount of their own prowesses and failings.
Processing raw materials is considered unclean, but is also fundamental to the local economy. 
Creating useful tools and objects, or works of art, is considered a higher-calling. 
Musical performance is popular, with several string instruments which are either drawn with a bow or struck with fingers being common. 
But one instrument is considered the peak of musical artistry; The Symphonia, requiring two people to play, and capable of uninterrupted music. 
Meanwhile a very similarly conceived but smaller and simpler instrument is considered that of beggars and degenerates; the Leier. 
Both instruments continually draw on strings by way of a hand-cranked wheel, the two person variant is usually accompanied with one or both players singing, while the lesser version is sometimes accompanied with other simple instruments like knee-cymbals.

\subsubsection{Current Events}
The Triduchy is undergoing an era of massive change, for generations little to no interaction with any of the lands beyond the great desert has occurred, but within the last decade all that has changed. 
Great foreign demand has grown for the wares of the Triduchy, and the Arsenic harvested from Lake Kos is of great importance in places where bronze is made. The metropolis had doubled in population over the last generation, mostly among the slaves of the Duchy of Kos. 
There have never been so many undead among the labourers before, and stress is high. It has been generations since the last time there were armies in the Three-Lakes Region, but there are those who call to raise them again, either for defence from without or within. 
The nearby Kalmic Confederacy has petitioned for the Triduchy to join, but have been rebuked thus far.
The Duchess of Kos is with child, and the father has not been declared publicly, but it is an open secret that her child has been sired by the Duke of Kara. 
The duchess and duke are close, and well known to conspire together on their larger goals. 
They appear to be biding their time before any announcements they plan to make. 
While as of late the region’s bards have found themselves being paid to sing songs and tell stories of Geis, someone is giving patronage intending to make myths of his return.

\subsubsection{The Duchy of Kos}
The Peoples of this region are poor, many of them slaves. Workers toil away in the muck of Lake Kos to eek out what meagre living they can. Great wealth is made by extracting precious metals from the dredge of the lake, but the process is deadly and kills many slaves. Living in the Duchy of Kos can be incredibly inexpensive, but the quality of life can also be incredibly poor. Buildings tend to range from mud and daub huts to tents.

\paragraph{The Duchess of Kos}
Neera Sandeater inherited the office from her uncle, after no more suitable heir was available. She was Thirty Eight at the time, three years have passed since then. She has proven popular among the people of the Duchy of Kos, and is known to appear in unseemly worksites to hand out food and blessings. She is considered less popular among the peoples of the other two duchies for this uncouthly charitable behavior. She is a strong-willed political figure, and since her rise to power the Duchy of Kos has seen a boom in population and economic output. She is sometimes compared to mythical Irso, but all-but entirely in hushed tones.

\paragraph{Daub the Humblest}
Among the Kosdredger’s Cult there is one man held in the highest of esteem, Daub is named the most humble as he has never claimed himself anyone’s leader. His insistence is that those who follow him do so entirely of their own accord. Even so, Daub is well known for representing the cult, and the poor of the Duchy of Kos by extension, in public settings. Furthermore Daub makes regular use of this power he wields as a bargaining chip at the negotiation table. To his followers, and much of the poorer peoples of the three-lakes region, Daub is something between a noble man and a living hero. While those who side opposite him during negotiations characterize Daub the Humblest as a sanctimonious fraud who happens to be in possession of an amiable personality and a shrewd ability to bargain.

\subsubsection{The Duchy of Mei}
Painters, carvers, weavers, and sculptors, all the best in the region reside or aspire to reside in the Duchy of Mei. The waters of Lake Mei are a deep abyss one may stare into to find clarity, and the shimmering light that comes from the water is known to inspire all manner of masterpieces. The markets of the Duchy of Mei are where some of the highest quality textiles and dyes in the world are to be found. Though with only seasonal access to potable water, and without much in the way of locally sourced foodstuffs, it can be very expensive to live in this region. Most buildings are brick of some kind, though mixed materials are often used in cheaper construction.

\paragraph{The Duke of Mei}
Roe Landhau is the Duke of Mei, he is Sixty Three years old, has four wives and many children. Roe is a balding and rotund but happy man, well liked by the people of the Duchy of Mei for his positive outlook and generous joviality. He is more than happy to take a back seat to the shared machinations of Duchess Neera and Duke Yorhm, as all three agree this is for the best.

\paragraph{Quinton Slakeclay}
Quinton is well regarded in his community and takes great pride in his place among They of Imperfect Flesh. Quinton is very humble, and by all accounts does not seem to even be aware of the gravity of his position. Nonetheless his word is taken very seriously by the cult beneath him, and to the most devout members Quinton is almost himself deified.

\subsubsection{The Duchy of Kara}
The lands surrounding Lake Kara are sacred, and only the most blessed of people may live there. The buildings are beautiful and opulent, the people wealthy. Unlike the other two duchies, the Duchy of Kara is centered around the grandest temple of the region. The temple is ancient, and was originally dedicated to Kara, but before even the foundation of the Triduchy it was sacked and rededicated to Yora. The older images of Kara were torn and broken, and new edifices depicting Yora replaced them, most of these new depictions feature a portent of Mei in the form of a dark halo. In myth, Yora sought the witness of Mei as she attempted to properly seal the earth around Kara. Luckily the formless dark god chose to aid the first woman, and thus eons were purchased during which mortals may live.

\paragraph{The Duke of Kara}
Yorhm Windshrike is a devout and ambitions holy man, he is Twenty Seven years old. When very young, Yorhm was taken from the slums around Lake Kos by a member of the church, and put to work with other poor orphans. Over the two decades since then, Yorhm has taken the loose and disorganized church that found him and turned it into a strictly structured and centralized force that directs influence all over the three-lakes region. Yorhm Windshrike is the leader of the Triduchy, as no-one who has the power to oppose him sees any value in doing so. He is a just, if often ruthless leader, and prosperity has befallen the kingdom in the few years since he took up the office of Duke.

\paragraph{Gnel, Yorkh, and Raka Glassmonger}
Gnel inherited the family business sixteen years ago, when he was only nine. His late father’s youngest widow, Yorkh, took up his hand in marriage. The Glassmonger family has made their fortune by exchanging partial ownership of glassware shops for the connections they can wield in selling the wares. The family business was in a state of minor decline for the next four years, and then the couple met Raka. At the time Yorkh was twenty one, and Raka was nineteen. Yorkh showed more tenacity for business than Gnel, and it was her idea to personally review the family holdings. Raka was the mistress of one of the standout workshops the family owned, in terms of production and efficiency. While the marriage is explained to the public as being between Gnel and Raka for simplicity, in reality Yorkh and Raka are the strongest tie in the marriage, and between them the Glassmonger business has grown stronger than ever before.

\subsubsection{Cults of the Three-Lakes Region}
\paragraph{The Meiyoran / Nakaran Cult}
Known by either name, or just as often referred to as The Church, this cult is the most powerful organized religion of the three-lakes region. While it’s members do not make up the majority of society, they do make up the largest religion within it. Yorhm Windstrike was voted in seven years ago as the head of this religion, which also makes him the Duke of Kara.

\paragraph{The Cult of Hyrspeech}
A very loose and non-centralized cult with extremely varied followers. Those who teach and disseminate knowledge of trades are considered by society and themselves to stand aside from the rest of those who practice said trades, as this spread of information is considered under the divine auspices of Hyr. While in no way organized, and without much in the way of collective meeting-places, Hyrsayers most likely make up the second most popular cult in the region. Some members of this cult exclusively deal in information, taking the form of librarians, storytellers, or brokers of scrolls and tablets. Certain Hyrsayers take pride in the practice of collecting and disseminating terrible truths, those who do often work with a necessary degree of secrecy. Some local bards consider themselves Hyrsayers, while other local bards often consider this first group pompous.

\paragraph{Kosdreger’s Cult}
Kosdredging is a derogatory term for making your living in the waters of Lake Kos. The Cult Members do not refer to themselves thusly, instead calling themselves some variation of Adherents to, or Followers of Fuug. This cult is popular among the poor labourers, including the undead, which is interesting as the cult has a distinctly anti-undead belief system. As well as the locally universal truth that Undeath is an unclean behavior, the Kosdredger’s Cult believes that an undead who exhibits strength of will while working has “Captured” the power of Fuug and that this situation is not desirable. Within the cult’s practices such an undead worker is expected to make full use of this misappropriated blessing and work themselves to a proper death, then and there. Beyond this, the cult is a powerful positive force in the worksites of the Duchy of Kos. Making collective efforts to plan around spawning booms of Kostic Eels, and attempting on occasion to bargain on behalf of the workers for better treatment.

\paragraph{The Few}
Some of those who make their living through forage have formed a cult centered around a particular concept: The world of the Three-Lakes Region is clearly the domain of The Many, not mortals. They leave lavish offerings of expensive meats when they take clay, believing this will result in greater forage in return. Followers of this cult are happy to subsist, and consider what passes for progress as foolhardy and avaricious.

\paragraph{They of Imperfect Flesh}
A cult popular with men of minor wealth, as well as some fathers and working men. The central tenets of the cult are that just like the Grimm Siblings before Them, They are creatures of imperfect leftover flesh. But They of Imperfect Flesh believe that unlike the mythical progenitors of mankind, They can strive to be of the form their flesh and blood can be. They aspire to be like Geis, a more perfect form of Mankind. They strive to be healthy and strong both physically and socially. Some of Them are tenaciously dogmatic and evangelical of Their beliefs, others simply consider the belief system a valuable tool in their way of life.