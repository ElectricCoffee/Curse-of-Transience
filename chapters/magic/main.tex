\chapter{Magic}
The world is about as magical as it is mundane.
Every living creature, every word uttered, every thing imaginable has traces of magic within.

In the region of Saoghaal, three different ``schools'' of magic exist: \textit{Bardery}, \textit{Sorcery}, and \textit{Thermomancy}.
Each of these three schools serve different purposes, require different rituals, and are fundamentally incompatible with each other.

\section{Bardery}
\textit{The sombre tones of a lute and a gentle voice fills the air.
The noise in the tavern dies down a bit, giving room for the songstress' heartfelt melody.
You slump back in your chair teary-eyed, throwing your hand of cards to the table.
The song is slowly clouding your mind, filling it with depressing imagery of a time you thought long forgotten.
Everything else around you seems to disappear---there's only you, the song, and your memories.}\\\\
The songs and stories of a skilled bard can be both powerful and enticing, and in the wrong hands outright dangerous.
Bards seem to have the uncanny ability to pull, tug, and pluck the heart-strings and nerves of every given individual in their vicinity.
Their words and songs laced with magic, able to inspire, enrage, and even heal their fellow teammates.

Bardic magic works by manipulating the emotions of people. 
Given enough skill and finesse, a talented bard can make anyone do anything by simply choosing the right song, act, or words.
Bards are entertainers at heart, but even the greatest performers can have ulterior motives---
next thing you know, the handsome young performer might be gone and your purse along with him!

\paragraph{Background} Bardery is said to have been invented by Sirens, who supposedly used their songs to lure sailors to an early watery grave.
Though they claim this to be nothing but a myth, Sirens remain the best natural users of bardic magic of all the known species in Saoghaal.

\paragraph{Catalyst} Musical instruments and the spoken word.
Better and more magical instruments are capable of augmenting the bardic songs, amplifying their magic potential.

\paragraph{Note} Unlike Sorcery, Bardery does not rely on any particular spells; 
any song, poem, story, or act that provokes a given emotion can be used for this sort of purpose.
Some bards make use of a songbooks, novels, or manuscripts to serve as a reminder of how certain songs or plays go, but most bards tend to perform by heart.

\section{Sorcery}

\paragraph{Background}

\paragraph{Catalyst} The magical essence stored within flora, fauna, and minerals.

\section{Thermomancy}

\paragraph{Background}

\paragraph{Catalyst}