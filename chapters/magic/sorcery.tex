\section{Sourcery}
\textit{A putrid smell of rotting sea life fills the room, as an elderly gentleman mashes something squishy in a mortar.
He rummages through his old leather purse, the sound of small vials clacking accompany his mellow murmur.
``Ahh, that's the one!'' He suddenly exclaims gleefully as he uncorks a small vial of yellow powder, pouring its contents into the mortar.
The gentleman starts chanting a spell as he waves his hand over the mortar, and a soft blue light emerges from within.
Soon after he pulls out a small spatula from his purse and scoops up a bit of the glowing blue cement-like substance, before applying it to the wound of his friend.
``This should patch you right up!'' he mutters as the wound begins to close before their eyes.}\\\\
Sourcery can best be described as a science, more than magic, and is often treated as such.
It closely resembles a mix of chemistry, alchemy, and incantations, and even the slightest mistakes in the incantation process can alter the power or effect of the resulting spell quite significantly.
\paragraph{Background}

\paragraph{Catalyst} The magical essence stored within flora, fauna, and minerals.
