\newpage\section{Sourcery}
\textit{A putrid smell of rotting sea life fills the room, as an elderly gentleman mashes something squishy in a mortar.
He rummages through his old leather purse, the sound of small vials clacking accompany his mellow murmur.
``Ahh, that's the one!'' He suddenly exclaims gleefully as he uncorks a small vial of yellow powder, pouring its contents into the mortar.
The gentleman starts chanting a spell as he waves his hand over the mortar, and a soft blue light emerges from within.
Soon after he pulls out a small spatula from his purse and scoops up a bit of the glowing blue cement-like substance, before applying it to the wound of his friend.
``This should patch you right up!'' he mutters as the wound begins to close before their eyes.}\\\\
Sourcery can best be described as a science, more than magic, and is often treated as such.
It closely resembles a mix of chemistry, alchemy, and incantations, and even the slightest mistakes in the incantation process can alter the power or effect of the resulting spell quite significantly.

Sourcerers are craftsmen and scholars who spend years of their lives researching and studying the art of conjuration.
The process of learning sourcery is long and arduous, and those who pursue it will often dedicate their entire lives to the cause.
The typical base-level sourcerer's training takes anywhere from 35 to 40 years to complete, with another 20 years for advanced training.
As such, no young sourcerers exist.

Sourcery also is capable of enchanting mundane items, a practice which the people of N\'angu\'o have used for great effect, 
having produced long-range transportation and broadcasting technology years ahead of the surrounding world by simply imbuing items with magic.

\paragraph{Background} Sourcery is said to have originated in N\'angu\'o eons ago, evidenced by the presence of ancient Sourcerer tools in former N\'angu\'o territories.

\paragraph{Catalyst} The magical essence stored within flora, fauna, and minerals.
